%!TEX root = ../report.tex

% 
% Objectives
% 

\section{Objectives (1pg) [NOT REVIEWED]}
\label{sec:objectives}% (fold)

%Clearly explain the project objectives.


The world population has grown at a faster and faster pace, which is causing overcrowding and lack of resources in various locations of the globe. For example in the Emirates, the economic boom supported by the oil business is creating lots of work oportunities, thus thousands of people from everywere in the world are moving there. However the country was not prepared to accommodate the growing population currently living there. 
To solve this problem of overpopulation that occurs there, and in other countries like China, new big cities are being built in the desert completely from scratch. An example of these cities is the city of maasdar in Abu Dhabi . This city designed by Foster and Partners and currently being built in the desert have a total of $6km^2$. Who is responsible for a project of this size can not afford to make a mistake in choosing the location or city setting.
To test these factors is necessary to create models , only in contrast to the relatively small cost of creating manually using traditional techniques , model a single building with some variations , nestad situations have to be modeled an entire city and even variations additional . This  design phase is therefore very costly in time and effort required to enable them to obtain the best possible result.

The objectives of this project are to study the existing approaches to Procedural Generation and provide a solutions to the kind of problems, like the new city in Abu Dabi stated above, through the implementation of mechanisms for procedural generation of architectural form, with application to the generation of large amounts of forms, such as of urban environments, buildings and ornaments efficiently and according to a particular architectural style models.

With the large amount of very different techniques, explained in Section~\ref{sub:procedural_modelling_techniques}, one important part of this work is to know where to apply each techinque to obtain better results.


%Implementation of these mechanisms in the Generative Design tool Rosetta.


%``As ferramentas de CAD existentes estão vocacionadas para uma utilização manual. Infelizmente, a produção manual de grandes quantidades de formas arquitectónicas complexas é muito morosa. A geração procedimental dessas formas é uma das abordagens que permite acelerar substancialmente esse processo. Esta abordagem consiste na construção algorítmica dessas formas, empregando Gramáticas de Formas, L-systems, Autómatos Celulares, etc. Nesta tese pretende a exploração destas técnicas e a sua implementação na ferramenta Rosetta. A avaliação poderá incluir geração de ambientes urbanos, de edifícios e de ornamentação de acordo com um determinado estilo arquitectónico.``

% section Objectives (end)