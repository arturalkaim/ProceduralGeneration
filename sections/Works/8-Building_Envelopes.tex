%!TEX root = ../../report.tex

Building Envelopes
\subsection{Building Envelopes [NOT DONE]} % (fold)
\label{sub:building_envelopes}

Sabri Gokmen in \cite{Gokmen2013} presents a way to create envelope systems for buildings. In this case he had a approach that was inspired in Gotheam morphology and leaf venetian patterns.

This article explains form as described by Johann Wolfgang von Goethe in the late eighteenth century. He started by working on annual plants through the work of Linneaus. Goethe considered the external properties of plants to be the result of an internal principle, idea that was influenced by the prior topological work by Linneaus that classified plants according to their physical characteristics.
For Goethe this external properties are not constant and change over time according to environmental conditions.

Forms in architecture are described as a ``topological entity following an overall schema or as a replication of an exiting type that appears fixed (Garcia, 2010)". It says that architects often work with topologies for various buildings and variations are achieved using topological operations. Adding to this is the idea of parametric design to create ``smooth variable systems". This parametric  top-down systems are able to control the overall behaviour of design and have been used to evaluate and adapt performance based approaches in design solutions. ``However this systems are However these systems are ineffective to provide a morphogenetic approach towards design."

Because morphology considers form as the result of a bottom-up process, it is a better method to model growth. Because the form is not defined from start and is the result of the growth process.

Goethe mixed this two ideas namely performance based and morphogenetic approaches. The parametric property is not throughout the system but applied in parts that grow with a morphogenetic approach.

Leaf venetian patterns:
There isn't one certain theory about the guiding principles that support the leaf patterns, there are many theories that try to explain it. One of them is called canalization theory that observes this patterns really as a distribution network, so it depends on the concentration/distributions of auxin producers to efficiently distribute auxin throughout the plant.

Computation of leaf venetian patterns:
The growth algorithm that is presented in this paper is based on the canalization theory. It starts by generating a density map that guides de distribution of auxin sources, that is the second step.
Finally it generates the leaf venetian patterns from the auxin source map.

The generation of the venetian patterns starts by setting root nodes that will be the points from which the patterns will start growing. Then ``at each time step the closest vein node to each auxin source will be defined. Then these nodes will grow towards the average direction influenced by the auxin sources".


% subsection building_envelopes (end)
