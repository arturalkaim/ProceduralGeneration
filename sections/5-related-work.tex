%!TEX root = ../report.tex

% 
% Related work
% 


\section{Related Work (~17pgs)}
\label{sec:related_work}
%\input{sections/Works/0-citation-template.tex}

As an active research field, there is a lot of work being done in this area. In this section I give an overview of the related work that has been carried out on procedural generation of cities. The generation of cities can be viewd as the sum of a few parts. The generation of terrains, road networks, buildings and other man made structures and cities in general.

The first step, the \emph{generation of terrains} is often left out of procedural generation of cities, because there is also lot of research focused only on this area, mainly to produce models of nature. In this work I will not evaluate specificaly this factor but make specific comments when relevant.

The \emph{road network} is a fundamental step for the generation of any city. It gives the city it's structure and defines the overall look. Most of the road networks are much different from each others and this makes the generalization dificult, but we can see some patterns when we analyze a road network from real cities. And this patterns are important for us if we want to have procedures that mimic this complex patterns. Some cities a have tightly structured grid network, like New York or a concentric radial pattern like Paris and others are purelly organic with the network presenting a nearlly random pattern. And we also have all the shades of grey in between.

Next step is the \emph{generation of building}. This is also not easy, since that there is no limit to the number of different buildings that can be modeled. Since buildings naturally are different from each other both in functionality and style, this is a complex problem to model. One widelly adopted solution is to use group buildings by functionality. The most common groups are \emph{Commercial}, \emph{Residential} and \emph{Industrial} buildings. After this we have to care only about modelling different styles adapted to each functional group. This part is also hard because this style can be anything. The architectural styles, as explained in the Section~\ref{sub:architectural_styles}, have influences from a lot of different sources. 

The last step is about the generation of the city itself, mainly join all the problems stated above and find a realistic layout for the city. Where to put each building is an important question that can have big impacts in the overall look of the models. Also how to group buildings and create neighborhoods, the function of the building is crucial, industrial buildings usualy are far from the residential ones, the same way that the comercial buildings are close together and also close to the residential areas.

\input{sections/Works/1-CityEngine.tex}

\input{sections/Works/2-Undiscovered-City.tex}

%!TEX root = ../../report.tex

\subsection{CityGen} % (fold)
\label{sec:citygen}
\cite{Kelly2008}

CityGen it's an interactive system that aims  to ``rapidly create the urban geometry typical of a modern city". The users can interact and control the generation process. The system, like others, is able to generate road networks that act as foundations to the model. It also can generate buildings but can not achieve the complexity and realism of other systems.


\subsubsection{Road Network} % (fold)
\label{ssub:road_network}


CityGen divided this problem in two steps. First the generation of the ``Primary Road Network", and after that, the ``Secondary Road Network". This two steps use different methods to generate the roads.
Undirected planar graphs are used to represent all roads. Two graphs for the Primary roads and one for each zone to store the secondary roads.


\emph{Primary Road Generation}

The primary road network uses two graphs, one high level graph that correspond directly to the primary road intersections. It represents the topological structure of the city by it's primary roads, and connections between them. The user is allowed to manipulate this high level graph, to change the high level structure ("topography of the primary road network") of the city .
There is also the low level graph that is generated from the other one, and defines the real path that the roads have in the terrain. It have the same nodes as the first graph and many more, that indicate the points on the terrain which the road passes.
To generate the low level graph it is used ``sampling, plotting and interpolation processes".


	

\begin{figure}[htbp]
	\centering
	\includegraphics[width=0.85\textwidth]{img/CityGen/RoadGraphs.png}
	\caption{The lighter graph is the High level graph, and the evolution to the darker Low-level graph }
	\label{fig:graphs}
\end{figure}

\emph{Secondary Road Generation}

The author defined city cells as districts, that are the areas of terrain that are enclosed by primary roads. The secondary road network is generated inside this cells using a growth based algorithm similar to the L-Systems technique as shown in Figure~\ref{fig:graphs2}.

\begin{figure}[htbp]
	\centering
	\includegraphics[width=0.85\textwidth]{img/CityGen/SecondaryRoadGrowth.png}
	\caption{}
	\label{fig:graphs2}
\end{figure}

% subsubsection road_network (end)


\subsubsection{Buildings} % (fold)
\label{ssub:buildings}



This system generates buildings also. Each building is created in lots that are identified after the extraction of enclosed regions, called blocks, from the secondary graph. Lots that don't have direct access to the roads are excluded. 


Based on the type of block the building footprints are created. After that building geometry is generated by extruding the footprint. The height of each is determined by a height parameter and a noise factor that can be also manipulated. A block is shown in the Figure~\ref{fig:primitiveShapes}, with only primitive shapes.

\begin{figure}[htbp]
	\centering
	\includegraphics[width=0.85\textwidth]{img/CityGen/BockPrimitiveShapes.png}
	\caption{Primitive Shape Buildings}
	\label{fig:primitiveShapes}
\end{figure}

With this primitive shapes, CityGen uses ``advanced materials with shaders to simulate additional geometry".

% subsubsection buildings (end)


% subsection citygen (end)


%!TEX root = ../../report.tex

\subsection{Inverse Design [NOT DONE]} % (fold)
\label{sub:inverse_design}


In \cite{Vanegas2009} it is presented a different solution from the others already presented.

It's described a framework that enables high level control of the modelling process. It ``provides a mechanism to interactively edit urban models". They apply inverse design to solve the problem of output control. From an existing model, the user can specify high level indicators that describe the desired output and the system change the underlying rules and parameters to get the result as close as possible to the desired.

As the Figure~\ref{fig:loop} shows, the user can change the ``low level" parameters and the ``high level" indicators to control the final output of the system.

\begin{figure}[htbp]
	\centering
	\includegraphics[width=0.95\textwidth]{img/Inverse_Design/TheLoop.PNG}
	\caption{System Pipeline \cite{Vanegas2009}}
	\label{fig:loop}
\end{figure}

This framework uses the indicators as a goal to optimize the parameters. To calculate the values for the parameters they used a version of Monte Carlo Markov Chains (or MCMC) and Resilient Back Propagation.


They implemented a urban procedural engine ``similar to previous city-level procedural modelling work". It was inspired by urban planners, that use the place types concept to represent coherent design patterns of buildings and streets. 


With this approach, the authors claim two main advantages, \emph{Abstraction} and \emph{Interactivity}. They argue that this solution allows urban planners and designers to work at a high level of abstraction, enabling users to manipulate place types, parameters and indicators to create the 3D model they want without wasting time with low level tasks as implementation of low level rules or parameter tuning.

At the same time this approach enables the users to interactively manipulate very complex indicator targets with a ``sophisticated enough" methodology to map target indicators to input low level parameters.


With their teilored version of MCMC and back-propagation they are able to support complex indicators ``enabling control beyond global shape, sush as by high-level semantics and indicators'', and by considering the procedural model as a black box there aren't any limitations to the used grammar.


% \subsubsection{Overview} % (fold)
% \label{ssub:overview}

% A system $P$ produces a geometry $G$ based on \emph{m} input parameter values $\Phi = \{\phi_1,\dots,\phi_m\}$ that is evaluated by an indicator measurement system $I$ which produces a set $\Gamma = \{\gamma_1,\dots,\gama_n\}$ of \emph{n} indicator values.


% % subsubsection overview (end)


% \subsubsection{Inverse Design} % (fold)
% \label{ssub:inverse_design}

% $\Gamma$ - Indicator Values

% $\Phi$ - Input Parameters

% $\Gamma*$ - Target Indicator Values

% $\Phi*$ - Target Parameter Values (specification is optional)


% % subsubsection inverse_design (end)


% \subsubsection{Urban Procedural Model} % (fold)
% \label{ssub:urban_procedural_model}



% subsubsection urban_procedural_model (end)


(\dots ?)

% subsection inverse_design (end)


%!TEX root = ../../report.tex

\subsection{CityBuilder} % (fold)
\label{sub:citybuilder}


CityBuilder is a system introduced by Watson et. al in \cite{Lechner2003} and \cite{Report2004} ``Procedural City Modeling" and ``Procedural Modeling of Land Use in Cities"}. 
It aims to be self automated to minimize necessary input, only needs the terrain description. Although it allows some other input from the user to give some interaction and control. 
To achieve that, it uses agent based simulation to create a system of agents and behaviours that can model specific entities of a city as developers, planning authorities and road builders. The set of rules for each agent is small to achieve a simple behaviour. With that, they what to make their ``model extendible not only in regard to the types of structures that are produced but also in describing the social and cultural influences prevalent in all cities."

\subsubsection{Road Network} % (fold)
\label{ssub:road_network}
The road network is designed by the agents based on the input terrain description wish will describe the height map and forbidden areas for roads, as water. 
The user can specify that the roads must follow a pre-established gridded pattern, or give the freedom to network to grow more organic. The image presents this two options and a combination of the two with a centre with a gridded pattern ant organic surroundings.


There are two types of road building agents, the extenders and the connectors.

The extenders search the area around exiting developments to look for areas that are not being served by any road to expand the current road network.

The connectors roam trough existing roads and sampling random points in the road network within a given radius. It tries to reach that point trough the road via a BSF. If it cannot reach or the distance needed is over a threshold value, the agent tries to create a road between these two points.


% subsubsection road_network (end)

\subsubsection{Buildings} % (fold)
\label{ssub:buildings}

This system does not develop buildings, but it's developer agents generate parcels and specify the use of the land. They can identify ``at least nine different types of land usages". They perform the role of urban developers that buy land, request planing permission, build and sell. They track the usage of their lands and specify the parameters to the buildings that can be build there.

% subsubsection buildings (end)

\subsubsection{The City} % (fold)
\label{ssub:the_city}

This system develop an evolving conceptual model of one city, that can represent the growth and evolution of a city through the time. 

% subsubsection the_city (end)

% subsection citybuilder (end)


%!TEX root = ../../report.tex

Building Envelopes
\subsection{Building Envelopes [NOT DONE]} % (fold)
\label{sub:building_envelopes}

Sabri Gokmen in \cite{Gokmen2013} presents a way to create envelope systems for buildings. In this case he had a approach that was inspired in Gotheam morphology and leaf venetian patterns.

This article explains form as described by Johann Wolfgang von Goethe in the late eighteenth century. He started by working on annual plants through the work of Linneaus. Goethe considered the external properties of plants to be the result of an internal principle, idea that was influenced by the prior topological work by Linneaus that classified plants according to their physical characteristics.
For Goethe this external properties are not constant and change over time according to environmental conditions.

Forms in architecture are described as a ``topological entity following an overall schema or as a replication of an exiting type that appears fixed (Garcia, 2010)". It says that architects often work with topologies for various buildings and variations are achieved using topological operations. Adding to this is the idea of parametric design to create ``smooth variable systems". This parametric  top-down systems are able to control the overall behaviour of design and have been used to evaluate and adapt performance based approaches in design solutions. ``However this systems are However these systems are ineffective to provide a morphogenetic approach towards design."

Because morphology considers form as the result of a bottom-up process, it is a better method to model growth. Because the form is not defined from start and is the result of the growth process.

Goethe mixed this two ideas namely performance based and morphogenetic approaches. The parametric property is not throughout the system but applied in parts that grow with a morphogenetic approach.

Leaf venetian patterns:
There isn't one certain theory about the guiding principles that support the leaf patterns, there are many theories that try to explain it. One of them is called canalization theory that observes this patterns really as a distribution network, so it depends on the concentration/distributions of auxin producers to efficiently distribute auxin throughout the plant.

Computation of leaf venetian patterns:
The growth algorithm that is presented in this paper is based on the canalization theory. It starts by generating a density map that guides de distribution of auxin sources, that is the second step.
Finally it generates the leaf venetian patterns from the auxin source map.

The generation of the venetian patterns starts by setting root nodes that will be the points from which the patterns will start growing. Then ``at each time step the closest vein node to each auxin source will be defined. Then these nodes will grow towards the average direction influenced by the auxin sources".


% subsection building_envelopes (end)


\subsection{Suicidator City Generator:} % (fold)
\label{sub:suicidator_city_generator}


http://cgchan.com/suicidator/index.php
Suicidator is an addon for the 3d application Blender.


“It varies the colors, materials and procedural elements enough to produce a reasonable approximation of a city at a distance, but realism is not a goal, and it does not operate on real data at all.” (http://vterrain.org/Culture/BldCity/Proc/ )

% subsection suicidator_city_generator_ (end)

\subsection{ghostTown:} % (fold)
\label{sub:ghosttown}

http://kilad.net/site/
http://www.kilad.net/GTForum/


It’s a script plugin for 3DS Max from Autodesk.
“Ghost Town is a script that procedurally generates cities and urban environments in only a few clicks, and features a number of options. Such as low or high poly buildings, road layouts, vehicles, trees, facades and an easy to use material and texture system.” (http://cgi.tutsplus.com/tutorials/create-a-detailed-city-with-3d-studio-max-ghost-town--cg-10090)

% subsection ghosttown_ (end)

\subsection{Skyscraper:} % (fold)
\label{sub:skyscraper}

http://www.skyscrapersim.com/index.shtml
Standalone


“Skyscraper aims to be a fully-featured, modular, 3D realtime building simulator, powered by the Scalable Building Simulator (SBS) engine. The main feature SBS provides is a very elaborate and realistic elevator simulator, but also simulates general building features such as walls, floors, stairs, shaftwork and more. Many more things are planned, including gaming support (single and network multiplayer), and a graphical building designer. Skyscraper is written in C++ and uses the OGRE graphics engine, Bullet for collisions and physics, FMOD for sound, the wxWidgets GUI library, and is multiplatform. The current versions aim for a future 2.0 release.” from the official site.

% subsection skyscraper (end)

\subsection{Blended Cities:} % (fold)
\label{sub:blended_cities}

http://jerome.le.chat.free.fr/index.php/en/city-engine/
Addon for Blender

“Blended Cities is an open-source city generator for Blender. it allows to create quickly a large amount of streets and buildings, with various shapes. B.C. fights against squared things : curved streets and odd or cylindric buildings can be created simply, so you can create old towns, not only modern cities.”

% subsection blended_cities_ (end)

%\input{sections/Works/0-citation-template.tex}


