%!TEX root = ../report.tex

% 
% Introduction
% 

\section{Introduction (2/3pgs)}
\label{sec:Introduction}

%General description of the problem and its context, current solutions, and road map of the project.

As technology evolves and people get new and more powerful devices, they want to take advantage of that with more detailed and complex contents to have more realistic experiences.
And this is observable in the graphic contents. With more de definition of the screens and the computational power of the machines beating records, the graphic content have to follow up that characteristics in quantity as well as in quality.  The issue is that the manual content generation takes a long work time from architects and designers to achieve this quality. 

The obvious answer to this problem is to contract more architects or designers to each project to increase the production, but experience have shown that this solution is not scalable, that means that double the number of architects or designers working in a project will not double their overall productivity. And this solution have a big impact on costs, that would take immediately out of the market new producers with less resources.

A solution for this problem is the use of generative design. That is a design method that is based on a programming approach which allows architects and designers to model complex shapes with significantly less effort. 

%Most computer-aided design (CAD) application provide progamming languages for generative design, programs writen in these languages have very limited portability and are not pedagogical and most of them are poorly designed or obsolete, reasons that create barriers to adherence to this approach by users that normally are not used to code \cite{ramos_et_al:OASIcs:2014:4565}.

There is an area of research that addresses this problem with \emph{Procedural Content Generation}, or \emph{Procedural Generation}. This have applications for example in the creation of large and/or complex scenarios for games and movies or the creation of models to use for simulation of cities. This examples involves the generation of large amounts of forms that would be impracticable with the manual approach to content creation.




% section section_name (end)

